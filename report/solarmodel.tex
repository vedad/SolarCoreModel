\documentclass[a4paper, 11pt, english]{article}
%
% Importering av pakker
%
\usepackage[utf8]{inputenc}
\usepackage[T1]{fontenc, url}
\usepackage[english]{babel}
\usepackage{textcomp}
\usepackage{amsmath, amssymb}
\usepackage{amsbsy, amsfonts}
\usepackage{graphicx, color}
\usepackage{parskip} % Space between paragraphs instead of indented text.
\usepackage{fancyhdr}
\usepackage{lastpage}
\usepackage{a4wide} % Smaller margins.
\usepackage{booktabs} % Functionality for tables.
\usepackage{lipsum} % Placeholder text
\usepackage{lmodern}
%
% Fancy captions
%
\usepackage[small,bf]{caption}
\usepackage{hyperref}
\usepackage{fancyhdr}
%
% Algorithms
%
\usepackage{algpseudocode}
\usepackage{algorithm}
%
% Subfigure functionality
%
\usepackage{caption}
\usepackage{subcaption}
%
% Parametere for inkludering av kode fra fil
%
\usepackage{listings}
%\lstset{language=python}
%\lstset{basicstyle=\ttfamily\small}
%\lstset{frame=single}
%\lstset{keywordstyle=\color{red}\bfseries}
%\lstset{commentstyle=\itshape\color{blue}}
\lstset{showspaces=false}
\lstset{showstringspaces=false}
\lstset{showtabs=false}
\lstset{breaklines}
\lstset{tabsize=4}
%
% Cool code listing
%
\lstdefinestyle{custompython}{
  belowcaptionskip=1\baselineskip,
  breaklines=true,
  frame=L,
  xleftmargin=\parindent,
  language=python,
  showstringspaces=false,
  basicstyle=\footnotesize\ttfamily,
  keywordstyle=\bfseries\color{red}\bfseries,
  commentstyle=\itshape\color{blue},
  identifierstyle=\color{black},
  stringstyle=\color{magenta},
}
\lstdefinestyle{customasm}{
  belowcaptionskip=1\baselineskip,
  frame=L,
  xleftmargin=\parindent,
  language=python,
  basicstyle=\footnotesize\ttfamily,
  commentstyle=\itshape\color{blue},
}
\lstset{escapechar=@,style=custompython}
%
% Pseudokode
%
\usepackage{xcolor}
\usepackage{listings}
\usepackage{caption}
\DeclareCaptionFont{white}{\color{white}}
\DeclareCaptionFormat{listing}{%
  \parbox{\textwidth}{\colorbox{gray}{\parbox{\textwidth}{#1#2#3}}\vskip-4pt}}
\captionsetup[lstlisting]{format=listing,labelfont=white,textfont=white}
\lstset{frame=lrb,xleftmargin=\fboxsep,xrightmargin=-\fboxsep}
%
% Definering av egne kommandoer og miljøer
%
\newcommand{\dd}[1]{\ \mathrm{d}#1} % 
\newcommand{\D}[1]{\ \mathrm{d}#1} % steps in integrals, ex: 4x \D{x} -> 4x dx
\newcommand{\f}[2]{\frac{#1}{#2}}
\newcommand{\beq}{\begin{equation*}}
\newcommand{\eeq}{\end{equation*}}

\newcommand{\refeq}[1]{(\textcolor{red}{\ref{eq:#1}})} % Red color when referencing equations.
\newcommand{\refig}[1]{\textcolor{blue}{\ref{fig:#1}}} % Blue color when referencing figures.
\newcommand{\reflst}[1]{(\textcolor{red}{\ref{lst:#1}})}
\newcommand{\reftab}[1]{\textcolor{blue}{\ref{tab:#1}}} % Blue color when referencing tables.

\newcommand{\HRule}{\rule{\linewidth}{0.5mm}}

\newcommand{\course}{AST3310 - Astrophysical plasma and stellar interiors}
\newcommand{\handIn}{Project 1}
\newcommand{\projectTitle}{Modeling the solar core}

%
% Page formatting
%
%\pagestyle{fancy}
%\lhead{Vedad Hodzic}
%\rhead{}
%\cfoot{ \thepage\ of \pageref{LastPage} }
%
% Navn og tittel
%

\author{Vedad Hodzic}

\begin{document}

\begin{titlepage}
    \thispagestyle{empty}
    \input{frontPage.tex}
\end{titlepage}

\begin{abstract}
	I here discuss the properties of the interior of the Sun based on models and
	simplifications addressed in ~\cite{stix} and ~\cite{gudiksen}. Numerical calculations
	are used to solve the governing equations of the interior of the Sun.
	One should expect the luminosity, position and mass to end at zero
	value simultaneously, while the temperature lies around $\sim15$ MK. It is found
	that the results heavily depend on initial physical properties, and in some cases on the
	mass step $\partial m$. Unexpectedly, it is likely the mass step has little to say
	given the right initial conditions.
\end{abstract}

\section{Introduction}

There are four differential equations that govern the internal structure of the Sun in our
model. We apply here numerical methods such as the Euler integration scheme to solve these
equations. This involves solving equations for nuclear fusion as well as ordinary
differential equations. We look at how the inner Sun changes based on different initial
parameters, and try to seek out a set of parameters that leads to the case where mass,
position and luminosity reach zero value simultaneously. 

\section{Theory}

The equations that regulate the physical properties of the interior of the Sun are
expressed as
%%%
\begin{align}
	\frac{\partial r}{\partial m} &= \frac{1}{4\pi r^2 \rho} \label{eq:drdm} \\
	\frac{\partial P}{\partial m} &= -\frac{Gm}{4\pi r^4} \label{eq:dPdm} \\
	\frac{\partial L}{\partial m} &= \varepsilon \label{eq:dLdm} \\
	\frac{\partial T}{\partial m} &= -\frac{3 \kappa L}{256 \pi^2 \sigma r^4 T^3}. 
	\label{eq:dTdm} 
\end{align}
%%%
The variable $\varepsilon$ in \refeq{dLdm} is the total energy generation per unit mass.
It is found by looking at the energy generation from nuclear reactions. It depends on the
abundancy of different elements, temperature and density. The variable $\kappa$ is the
opacity, which is an average of frequency of photons.
The pressure in eq. \refeq{dPdm} is a sum of the gas pressure $P_{\mathrm{G}}$, and the radiative
pressure $P_{\mathrm{R}}$. 
\begin{align}
	P &= P_{\mathrm{G}} + P_{\mathrm{R}} \nonumber \\
	P &= \frac{\rho}{\mu m_{\mathrm{u}}} kT + \frac{a}{3}T^4 \nonumber \\
	\Rightarrow \rho &= \frac{\mu m_{\mathrm{u}}}{kT} \left( P - \frac{a}{3}T^4 \right),
	\label{eq:density}
\end{align}
which yields the density derived from the equation of state. Here, $m_{\mathrm{u}}$ is the
atomic mass unit and $k$ is the Boltzmann
constant. The constant $a$ is defined as $a = 4\sigma / c$, where $\sigma$ is the
Stefan-Boltzmann constant, and $c$ is the speed of light. The average molecular weight
$\mu$ is found by
\begin{align}
	\mu = \frac{1}{\mu}\frac{\rho}{n_{\mathrm{tot}}}.
	\label{eq:mu}
\end{align}
We can find the total particle density $n_{\mathrm{tot}}$ with
\begin{align*}
	n_{\mathrm{tot}} &= n_X + n_Y + n_Z + n_e \\
	&= \frac{X\rho}{m_{\mathrm{u}}} + \frac{Y\rho}{4m_{\mathrm{u}}} +
	\frac{Z\rho}{Am_{\mathrm{u}}} + \left( \frac{X\rho}{m_{\mathrm{u}}} +
	\frac{2Y\rho}{4m_{\mathrm{u}}} + \frac{Z\rho}{2m_{\mathrm{u}}} \right),
\end{align*}
where $A$ is the average atomic weight of the heavier elements present, which we assume to
be $A = 7$. This assumption is based on that we only know of two heavier elements present,
which are $^{7}\mathrm{Be}$ and $^{7}\mathrm{Li}$. We then get
\begin{align*}
	\mu &= \frac{\rho m_{\mathrm{u}}}{\rho m_{\mathrm{u}}} \left( \frac{1}{2X +
		\frac{3}{4}Y + \frac{9}{14}Z} \right) \\
		&= \frac{1}{2X +\frac{3}{4}Y + \frac{9}{14}Z},
\end{align*}
assuming all elements are fully ionised.

The total energy generation per unit mass $\varepsilon$, is found by
\begin{align}
	\varepsilon = \sum Q'_{ik} r_{ik},
	\label{eq:epsilon}
\end{align}
where $i,k$ represents two elements, $Q'_{ik}$ is the energy released from the fusion of
two elements, $r_{ik}$ is the reaction rates per unit mass for two elements. The energies
$Q'_{ik}$ from the pp chains are listed in ~\cite[p.~39, Table~2.1]{stix}. The reaction
rates per unit mass is defined by
\begin{align}
	r_{ik} = \frac{n_i n_k}{\rho(1 + \delta_{ik})} \lambda_{ik},
	\label{eq:r}
\end{align}
where $n_i,n_k$ is the number density for an element, $\delta_{ik}$ is the Kronecker delta and
$\lambda_{ik}$ is the reaction rate of a fusion. The number density of an element is
easily defined as
\begin{align}
	n = \frac{\rho \chi_a}{am_{\mathrm{u}}},
	\label{eq:n}
\end{align}
where $\chi$ is the number fraction of an element, and $a$ is the atomic number of the
element. We denote $X, Y, Z$ to be the number fractions of hydrogen, helium and heavier
metals, respectively. Finally, the reaction rates $\lambda_{ik}$ for two elements
$i,k$ can be found in ~\cite[p.~46, Table~2.3]{stix}.


\section{Algorithm}
\subsection{Simplifications}

In order to make a simple model I needed some simplifications. A list of
assumptions and simplifications follows.
\begin{itemize}
	\item[\textbullet] There is no change in the composition of elements as a function of
		radius. 
	\item[\textbullet] I assume there is no change in the density of deuterium, so that
		the rate of destruction of deuterium is the same as the production, and that any
		reaction involving deuterium happens instantaneous.
	\item[\textbullet] All nuclear energy comes from the three PP-chains. I have not
		included the CNO cycle.
	\item[\textbullet] I assume all elements to be fully inonised.
\end{itemize}

\subsection{Units}

Given that $\kappa$ is in units of $\left[\mathrm{cm}^2  \ \mathrm{g}^{-1}\right]$,
$\lambda$ in units of $\left[\mathrm{cm}^3 \ \mathrm{s}^-1 \right]$ and so on,
I chose to adapt the CGS unit system (centimetre-gram-second). Table \reftab{units} shows
an overview of the differences.
\begin{table}
	\centering
	\begin{tabular*}{\textwidth}{p{0.25\textwidth}@{\extracolsep{\fill}}p{0.15\textwidth}p{0.15\textwidth}p{0.15\textwidth}p{0.15\textwidth}}
		\toprule
		\toprule
		& \multicolumn{4}{c}{Unit system} \\
		\cmidrule{2-5}
		& \multicolumn{2}{c}{CGS} & \multicolumn{2}{c}{SI} \\
		\cmidrule{2-3}
		\cmidrule{4-5}
		Physical property & Unit name & Unit abbr. & Unit name & Unit abbr. \\
		\midrule
		Length & centimetre & cm & metre & m \\
		Weight & gram & g & kilogram & kg \\
		Time & second & s & second & s \\
		Temperature & kelvin & K & kelvin & K \\
		Energy & erg & erg & joule & J \\
		Pressure & barye & Ba & pascal & Pa \\
		\bottomrule
	\end{tabular*}
	\caption{Overview of the difference between CGS units and SI units.}
	\label{tab:units} 
\end{table}


\subsection{Structure}

We see from eq. \refeq{dTdm} that $\partial T / \partial m$ depends on the opacity,
$\kappa$. A table of opacities that correspond to different values of temperature and
density has been provided. I wrote a function that reads the file and stores values of
$T$, $R$ and $\kappa$ in separate arrays. Here, $R = R(T, \rho) = \rho / T_6$. The
function compares the table temperature with the
actual temperature, and the same with the variable $R$. It returns the $\kappa$ with
table values of $T$ and $R$ that most closely resembles the present values.

Next, I wrote a function that calculates the energy generation per mass unit from nuclear
reactions. This means entering the energy releases $Q'_{ik}$ from ~\cite[p.~39,
Table~2.1]{stix}, the reaction rates $\lambda_{ik}$ from ~\cite[p.~46,
Table~2.3]{stix} and finding the number densities $n$ for all particles that are involved
in a nuclear reaction. Given the number fractions $X,Y,Z$ and sub-fractions of these
for different elements, I was able to calculate the energy generation per mass by using
eq. \refeq{epsilon}, \refeq{r} and \refeq{n}.

I have a function that calculates the density at present time, given the temperature
$T$ and total pressure $P$. This is found from the equation of state, as given in eq.
\refeq{density}. This required me to calculate the average molecular weight $\mu$,
shown in eq. \refeq{mu}.

Further, I have four functions that return the right-hand side in eq. \refeq{drdm},
\refeq{dPdm}, \refeq{dLdm} and \refeq{dTdm}. These are called upon in another function
that integrates all the equations. I implemented a crude adaptive step system, where I
changed the mass step gradually while integrating further into the Sun.

\subsection{The Euler integration scheme}

For simplicity, I have chosen the Euler integration scheme to integrate the differential
equations. Listing \reflst{integration} shows how it is carried out.
\belowcaptionskip=-10pt
\begin{lstlisting}[label=lst:integration,caption=Euler integration loop]	
for i in range(n-1):
	r[i+1] = r[i] - drdm * dm
	P[i+1] = P[i] - dPdm * dm
	L[i+1] = L[i] - dLdm * dm
	T[i+1] = T[i] - dTdm * dm
	m[i+1] = m[i] - dm
\end{lstlisting}
The Euler method advances one step by adding the previous step value to the current value
of the right-hand side of the differential equations, multiplied by the mass step.
Unfortunately, the scheme carries a local truncation error proportional to the step size
squared, and a global truncation error that is proportional to the step size.

\subsection{Initial conditions}
Finally, we need a set of initial conditions to initiate the calculations. These are shown
in Table \reftab{initcond} expressed in the CGS unit system (see Table \reftab{units}).
\begin{table}
	\centering
	\begin{tabular*}{\textwidth}{p{2cm}@{\extracolsep{\fill}}p{2cm}p{2cm}p{2cm}}
		\toprule
		\toprule
		\multicolumn{2}{c}{Physical properties} & \multicolumn{2}{c}{Element abundancies}\\
		\cmidrule{1-2}
		\cmidrule{3-4}
		Parameter & Initial value & Element & Initial value \\
		\midrule
		$ L_0 $ & $ L_{\odot} $ & $X$ & 0.7 \\
		$ R_0 $ & $ 0.5R_{\odot}$ & $Y_3$ & $10^{-10}$ \\
		$ M_0 $ & $ 0.7M_{\odot}$ & $Y$ & 0.29 \\
		$\rho_0$ & $1$ g $\mathrm{cm}^{-3}$ & $Z$ & 0.01 \\
		$ T_0 $ & $ 10^5$ K & $ Z_{^{7}{\mathrm{Li}}} $ & $10^{-5} $ \\
		$ P_0 $ & $10^{12} $ Ba & $ Z_{^{7}{\mathrm{Be}}} $ & $10^{-5}$ \\
		\bottomrule
	\end{tabular*}
	\caption{Table that shows the initial physical parameters needed in order to initiate the
	calculations.}
	\label{tab:initcond}
\end{table}


\section{Results}

\subsection{Using original initial values}
Carrying out the integrations using the initial conditions shown in Table
\reftab{initcond}, we get the results shown in Figure \refig{init_results}. As we see from
these figures, the luminosity reaches zero before the position does, while we have
integrated through
just under $0.2 \ \%$ of the initial mass. Figure \refig{temperature_1} shows
a sudden leap in temperature to start with. This indicates that the initial temperature is
too low relative to the initial pressure and density. The equation of state tries to
compensate for this, which leads to a violent increase in temperature. The physical
parameter that changes abruptly is \emph{probably} determined by the order the differential
equations are solved. Had we solved them in another order, we would probably see a change.
\begin{figure}[htpb]
	\begin{subfigure}{0.49\textwidth}
		\includegraphics[width=\linewidth]{figures/position_1.eps}
		\caption{Position}
		\label{fig:position_1}
	\end{subfigure}\hfill
	\begin{subfigure}{0.49\textwidth}
		\includegraphics[width=\linewidth]{figures/pressure_1.eps}
		\caption{Pressure}
		\label{fig:pressure_1}
	\end{subfigure}\hfill
	\vspace{0.35cm}
	\begin{subfigure}{0.49\textwidth}
		\includegraphics[width=\linewidth]{figures/luminosity_1.eps}
		\caption{Luminosity}
		\label{fig:luminosity_1}
	\end{subfigure}\hfill
	\begin{subfigure}{0.49\textwidth}
		\includegraphics[width=\linewidth]{figures/temperature_1.eps}
		\caption{Temperature}
		\label{fig:temperature_1}
	\end{subfigure}
	\vspace{0.2cm}
	\caption{Showing results of solving the differential equations using the initial
		parameters as described in Table \reftab{initcond}.}
	\label{fig:init_results}
\end{figure}
%

\subsection{On the effect of changing $\partial m$}

For some sets of initial conditions, it is important to choose a low $\partial m$ to
prevent the luminosity or temperature from increasing too quickly. However, this is highly
time consuming. I chose to adapt a very small (of order $10^{-10}$ of the initial mass)
step in
the beginning, while gradually increasing it as we work our way into the core. I noticed
however that given some initial conditions, I could choose $\partial m$ to be as big as
$1/100$ of the original mass, and still get identical results as I did in e.g. Figure
\refig{T1e7_P1e16}. I can only draw the conclusion that $\partial m$ needs to be very
small if we start with some physical parameters that are too low (or high) compared to
others, as in the case with Figure \refig{temperature_1}.

\subsection{Decreasing the amount of $^{7}\mathrm{Li}$ and $^{7}\mathrm{Be}$}

As a solution to the temperature problem, we try to lower the number fraction of
$^{7}\mathrm{Li}$ and $^{7}\mathrm{Be}$. We set $Z_{^{7}\mathrm{Li}} = Z_{^{7}\mathrm{Be}}
= 10^{-13}$. This should lower the energy generation, thus the temperature through the
luminosity. The results are shown in Figure \refig{smallZ}.
%
\begin{figure}[htpb]
	\begin{subfigure}{0.49\textwidth}
		\includegraphics[width=\linewidth]{figures/position_smallZ.eps}
		\caption{Position}
		\label{fig:position_smallZ}
	\end{subfigure}\hfill
	\begin{subfigure}{0.49\textwidth}
		\includegraphics[width=\linewidth]{figures/pressure_smallZ.eps}
		\caption{Pressure}
		\label{fig:pressure_smallZ}
	\end{subfigure}\hfill
	\vspace{0.35cm}
	\begin{subfigure}{0.49\textwidth}
		\includegraphics[width=\linewidth]{figures/luminosity_smallZ.eps}
		\caption{Luminosity}
		\label{fig:luminosity_smallZ}
	\end{subfigure}\hfill
	\begin{subfigure}{0.49\textwidth}
		\includegraphics[width=\linewidth]{figures/temperature_smallZ.eps}
		\caption{Temperature}
		\label{fig:temperature_smallZ}
	\end{subfigure}
	\vspace{0.2cm}
	\caption{Integration results by using $Z_{^{7}\mathrm{Li}} = Z_{^{7}\mathrm{Be}}
	= 10^{-13}$.}
	\label{fig:smallZ}
\end{figure}
%
We already see a change, especially in the luminosity and temperature. Figure
\refig{luminosity_smallZ} now shows how the luminosity is close to constant to about
$10$ \% of the initial mass. Then it suddenly dips exponentially, reaching zero value at
about $82 \ \%$ of $M_0$. Again, the luminosity decreases much faster than the position and
mass. Figure \refig{position_smallZ} shows the position to be near $80$ \% into the core.
The temperature in Figure \refig{temperature_smallZ} shows a steadier increase to about
$15 \times 10^6$ K, which sounds reasonable as we approach the core. However, we see an
abrupt increase in the pressure. It rises higher than a factor of $10^5$ of the initial
pressure. We increase the initial pressure to see if it stabilises our results further.

\begin{figure}[htpb]
	\begin{subfigure}{0.49\textwidth}
		\includegraphics[width=\linewidth]{figures/position_T1e7_P1e16.eps}
		\caption{Position}
		\label{fig:position_T1e7_P1e16}
	\end{subfigure}\hfill
	\begin{subfigure}{0.49\textwidth}
		\includegraphics[width=\linewidth]{figures/pressure_T1e7_P1e16.eps}
		\caption{Pressure}
		\label{fig:pressure_T1e7_P1e16}
	\end{subfigure}\hfill
	\vspace{0.35cm}
	\begin{subfigure}{0.49\textwidth}
		\includegraphics[width=\linewidth]{figures/luminosity_T1e7_P1e16.eps}
		\caption{Luminosity}
		\label{fig:luminosity_T1e7_P1e16}
	\end{subfigure}\hfill
	\begin{subfigure}{0.49\textwidth}
		\includegraphics[width=\linewidth]{figures/temperature_T1e7_P1e16.eps}
		\caption{Temperature}
		\label{fig:temperature_T1e7_P1e16}
	\end{subfigure}
	\vspace{0.2cm}
	\caption{Integration results by increasing both the initial pressure and temperature.}
	\label{fig:T1e7_P1e16}
\end{figure}

\subsection{Increasing initial temperature and pressure}
Increasing the pressure by a factor $10^4$ of the original value, and the temperature by a
factor of $10^2$, yields Figure \refig{T1e7_P1e16}. We immediately notice that this time,
it is the mass $m$ that reaches zero. However, the luminosity and position stop at
$73 \ \%$ and $55 \ \%$, respectively. We also notice from Figure
\refig{temperature_T1e7_P1e16} that the temperature now increases faster towards the end,
instead of in the beginning as we have had until now. This is probably more likely than
the latter. This is the closest I have gotten to having the position, luminosity and mass
reach zero values simultaneously. There might be a combination of $\rho_0, P_0, T_0$ that
makes this possible, but I have not found it.

\subsection{Density and energy generation}
It could be interesting to see how the density and energy generation behave for a given
set of initial values. We continue with the most recent initial values and have Figure
\refig{density_epsilon}.
%
\begin{figure}[htpb]
	\centering
	\begin{subfigure}{0.49\textwidth}
		\includegraphics[width=\linewidth]{figures/density_T1e7_P1e16.eps}
		\caption{Density}
		\label{fig:density_T1e7_P1e16}
	\end{subfigure}\hfill
	\begin{subfigure}{0.49\textwidth}
		\includegraphics[width=\linewidth]{figures/energy_T1e7_P1e16.eps}
		\caption{Energy generation per mass}
		\label{fig:epsilon_T1e7_P1e16}
	\end{subfigure}
	\vspace{0.2cm}
	\caption{Figures that show how the density and energy generation behave as function of
		mass, given the same initial conditions as in Figure \refig{T1e7_P1e16}.}
	\label{fig:density_epsilon}
\end{figure}
%
We see that the density increases by a factor of 1.5. What is noticable however, is it how
it decreases again towards the end, making it have the same value for different masses.
This does not represent the physical situation, as the density should increase with
increasing pressure and temperature without dropping again. We see the energy generation
starts at a low value, but gradually increases as the temperature increases towards the core.
This makes sense as it is strongly dependent on the temperature and density, where both
increase towards the center.

\newpage
\section{The code}
\lstinputlisting[caption=The program that solves the problem., style=customasm]{../solarCore.py}

\begin{thebibliography}{9}
		\bibitem{stix} 
		Michael Stix,
		\emph{The Sun}.
		Springer, New York,
		2nd Edition,
		2002.

		\bibitem{gudiksen}
		Boris Gudiksen
		\emph{AST3310: Astrophysical plasma and stellar interiors}.
		2014.

\end{thebibliography}


\end{document}

