\documentclass[a4paper, 11pt, english]{article}
%
% Importering av pakker
%
\usepackage[utf8]{inputenc}
\usepackage[T1]{fontenc, url}
\usepackage[english]{babel}
\usepackage{textcomp}
\usepackage{amsmath, amssymb}
\usepackage{amsbsy, amsfonts}
\usepackage{graphicx, color}
\usepackage{parskip} % Space between paragraphs instead of indented text.
\usepackage{fancyhdr}
\usepackage{lastpage}
\usepackage{a4wide} % Smaller margins.
\usepackage{booktabs} % Functionality for tables.
\usepackage{lipsum} % Placeholder text
%
% Fancy captions
%
\usepackage[small,bf]{caption}
\usepackage{hyperref}
\usepackage{fancyhdr}
%
% Algorithms
%
\usepackage{algpseudocode}
\usepackage{algorithm}
%
% Subfigure functionality
%
\usepackage{caption}
\usepackage{subcaption}
%
% Parametere for inkludering av kode fra fil
%
\usepackage{listings}
\lstset{language=python}
\lstset{basicstyle=\ttfamily\small}
\lstset{frame=single}
\lstset{keywordstyle=\color{red}\bfseries}
\lstset{commentstyle=\itshape\color{blue}}
\lstset{showspaces=false}
\lstset{showstringspaces=false}
\lstset{showtabs=false}
\lstset{breaklines}
\lstset{tabsize=4}
%
% Definering av egne kommandoer og miljøer
%
\newcommand{\dd}[1]{\ \mathrm{d}#1} % 
\newcommand{\D}[1]{\ \mathrm{d}#1} % steps in integrals, ex: 4x \D{x} -> 4x dx
\newcommand{\f}[2]{\frac{#1}{#2}}
\newcommand{\beq}{\begin{equation*}}
\newcommand{\eeq}{\end{equation*}}

\newcommand{\refeq}[1]{(\textcolor{red}{\ref{eq:#1}})} % Red color when referencing equations.
\newcommand{\refig}[1]{\textcolor{blue}{\ref{fig:#1}}} % Blue color when referencing figures.
\newcommand{\reflst}[1]{(\textcolor{red}{\ref{lst:#1}})}
\newcommand{\reftab}[1]{\textcolor{blue}{\ref{tab:#1}}} % Blue color when referencing tables.

\newcommand{\HRule}{\rule{\linewidth}{0.5mm}}

\newcommand{\course}{AST3310 - Astrophysical plasma and stellar interiors}
\newcommand{\handIn}{Project 1}
\newcommand{\projectTitle}{Modeling the solar core}

%
% Page formatting
%
%\pagestyle{fancy}
%\lhead{Vedad Hodzic}
%\rhead{}
%\cfoot{ \thepage\ of \pageref{LastPage} }
%
% Navn og tittel
%

\author{Vedad Hodzic}

\begin{document}

\begin{titlepage}
    \thispagestyle{empty}
    % Explanation of \\[]-syntax:
% In the square brackets you can define a certain vertical space
% which is inserted in the text. Convenient for fancy formatting.

\begin{center}
    \LARGE University of Oslo\\[1.5cm]
	\Large \course \\ 
	\Large \handIn \\ [0.5cm]
    \HRule \\[0.4cm]

    % Title of project
    { \huge \bfseries \projectTitle \\[0.4cm] }

    \HRule \\[1.5cm]

    % Spaces between lines are there for line breaks

	\large Vedad Hodzic

    \vfill

    {\large \today}
\end{center}

\end{titlepage}

\begin{abstract}
	I here discuss the properties of the interior of the Sun based on models and
	simplifications addressed in ~\cite{stix} and ~\cite{gudiksen}. Numerical calculations
	are used to solve the governing equations of the interior of the Sun. Using the 
	simplifications, one should expect the temperature to end at $\sim15$ MK. It is found
	that the results heavily depends on initial properties, and in particular the mass
	step $\partial m$.
\end{abstract}

\section{Introduction}

\section{Theory}

The equations that regulate the physical properties of the interior of the Sun are
expressed as
%%%
\begin{align}
	\frac{\partial r}{\partial m} &= \frac{1}{4\pi r^2 \rho} \label{eq:drdm} \\
	\frac{\partial P}{\partial m} &= -\frac{Gm}{4\pi r^4} \label{eq:dPdm} \\
	\frac{\partial L}{\partial m} &= \varepsilon \label{eq:dLdm} \\
	\frac{\partial T}{\partial m} &= -\frac{3 \kappa L}{256 \pi^2 \sigma r^4 T^3}. 
	\label{eq:dTdm} 
\end{align}
%%%
The variable $\varepsilon$ in \refeq{dLdm} is the total energy generation per unit mass.
It is found by looking at the energy generation from nuclear reactions. It depends on the
abundancy of different elements, temperature and density. The variable $\kappa$ is the
opacity, which is an average of frequency of photons.
The pressure in eq. \refeq{dPdm} is a sum of the gas pressure $P_{\mathrm{G}}$, and the radiative
pressure $P_{\mathrm{R}}$. 
\begin{align}
	P &= P_{\mathrm{G}} + P_{\mathrm{R}} \nonumber \\
	P &= \frac{\rho}{\mu m_{\mathrm{u}}} kT + \frac{a}{3}T^4 \nonumber \\
	\Rightarrow \rho &= \frac{\mu m_{\mathrm{u}}}{kT} \left( P - \frac{a}{3}T^4 \right),
	\label{eq:density}
\end{align}
which yields the density derived from the equation of state. Here, $\mu$ is the average
molecular weight, $m_{\mathrm{u}}$ is the atomic mass unit and $k$ is the Boltzmann
constant. The constant $a$ is defined as $a = 4\sigma / c$, where $\sigma$ is the
Stefan-Boltzmann constant, and $c$ is the speed of light.

The total energy generation per unit mass $\varepsilon$, is found by
\begin{align}
	\varepsilon = \sum Q'_{ik} r_{ik},
	\label{eq:epsilon}
\end{align}
where $i,k$ represents two elements, $Q'_{ik}$ is the energy released from the fusion of
two elements, $r_{ik}$ is the reaction rates per unit mass for two elements. The energies
$Q'_{ik}$ from the pp chains are listed in ~\cite[p.~39, Table~2.1]{stix}. The reaction
rates per unit mass is defined by
\begin{align}
	r_{ik} = \frac{n_i n_k}{\rho(1 + \delta_{ik})} \lambda_{ik},
	\label{eq:r}
\end{align}
where $n_i,n_k$ is the number density for an element, $\delta_{ik}$ is the Kronecker delta and
$\lambda_{ik}$ is the reaction rate of a fusion. The number density of an element is
easily defined as
\begin{align}
	n = \frac{\rho \chi_a}{am_{\mathrm{u}}},
	\label{eq:n}
\end{align}
where $\chi$ is the number fraction of an element, and $a$ is the atomic number of the
element. We denote $X, Y, Z$ to be the number fractions of hydrogen, helium and heavier
metals, respectively. Finally, the reaction rates $\lambda_{ik}$ for two elements
$i,k$ can be found in ~\cite[p.~46, Table~2.3]{stix}.


\section{The code}
\subsection{Simplifications}

In order to make a simple model I needed some simplifications. A list of
assumptions and simplifications follows.
\begin{itemize}
	\item[\textbullet] There is no change in the composition of elements as a function of
		radius. 
	\item[\textbullet] I assume there is no change in the density of deuterium, so that
		the rate of destruction of deuterium is the same as the production, and that any
		reaction involving deuterium happens instantaneous.
	\item[\textbullet] All nuclear energy comes from the three PP-chains. I have not
		included the CNO cycle.
	\item[\textbullet] I assume all elements to be fully inonised.
\end{itemize}

\subsection{Units}

Given that $\kappa$ is in units of $\left[\mathrm{cm}^2  \ \mathrm{g}^{-1}\right]$,
$\lambda$ in units of $\left[\mathrm{cm}^3 \ \mathrm{s}^-1 \right]$ and so on,
I chose to adapt the CGS unit system (centimetre-gram-second). Table \reftab{units} shows
an overview of the differences.
\begin{table}
	\centering
	\begin{tabular*}{\textwidth}{p{0.25\textwidth}@{\extracolsep{\fill}}p{0.15\textwidth}p{0.15\textwidth}p{0.15\textwidth}p{0.15\textwidth}}
		\toprule
		\toprule
		& \multicolumn{4}{c}{Unit system} \\
		\cmidrule{2-5}
		& \multicolumn{2}{c}{CGS} & \multicolumn{2}{c}{SI} \\
		\cmidrule{2-3}
		\cmidrule{4-5}
		Physical property & Unit name & Unit abbr. & Unit name & Unit abbr. \\
		\midrule
		Length & centimetre & cm & metre & m \\
		Weight & gram & g & kilogram & kg \\
		Time & second & s & second & s \\
		Temperature & kelvin & K & kelvin & K \\
		Energy & erg & erg & joule & J \\
		Pressure & barye & Ba & pascal & Pa \\
		\bottomrule
	\end{tabular*}
	\caption{Overview of the difference between CGS units and SI units.}
	\label{tab:units} 
\end{table}


\subsection{Structure}

We see from eq. \refeq{dTdm} that $\partial T / \partial m$ depends on the opacity,
$\kappa$. A table of opacities that correspond to different values of temperature and
density has been provided. I wrote a function that reads the file and stores values of
$T$, $R$ and $\kappa$ in separate arrays. Here, $R = R(T, \rho) = \rho / T_6$. The
function compares the table temperature with the
actual temperature, and the same with the variable $R$. It returns the $\kappa$ with
table values of $T$ and $R$ that most closely resembles the present values.

Next, I wrote a function that calculates the energy generation per mass unit from nuclear
reactions. This means entering the energy releases $Q'_{ik}$ from ~\cite[p.~39,
Table~2.1]{stix}, the reaction rates $\lambda_{ik}$ from ~\cite[p.~46,
Table~2.3]{stix} and finding the number densities $n$ for all particles that are involved
in a nuclear reaction. Given the number fractions $X,Y,Z$ and sub-fractions of these
for different elements, I was able to calculate the energy generation per mass by using
eq. \refeq{epsilon}, \refeq{r} and \refeq{n}.

I have a function that calculates the density at present time, given the temperature
$T$ and total pressure $P$. This is found from the equation of state, as given in eq.
\refeq{density}. This required me to calculate the molecular weight, $\mu$. It is given on
the form
\begin{align*}
	\mu = \frac{1}{\mu}\frac{\rho}{n_{\mathrm{tot}}}.
\end{align*}
We can find the total particle density with
\begin{align*}
	n_{\mathrm{tot}} &= n_X + n_Y + n_Z + n_e \\
	&= \frac{X\rho}{m_{\mathrm{u}}} + \frac{Y\rho}{4m_{\mathrm{u}}} +
	\frac{Z\rho}{Am_{\mathrm{u}}} + \left( \frac{X\rho}{m_{\mathrm{u}}} +
	\frac{2Y\rho}{4m_{\mathrm{u}}} + \frac{Z\rho}{2m_{\mathrm{u}}} \right),
\end{align*}
where $A$ is the average atomic weight of the heavier elements present, which we assume to
be $A = 7$. This assumption is based on that we only know of two heavier elements present,
which are $^{7}\mathrm{Be}$ and $^{7}\mathrm{Li}$. We then get
\begin{align*}
	\mu &= \frac{\rho m_{\mathrm{u}}}{\rho m_{\mathrm{u}}} \left( \frac{1}{2X +
		\frac{3}{4}Y + \frac{9}{14}Z} \right) \\
		&= \frac{1}{2X +\frac{3}{4}Y + \frac{9}{14}Z}.
\end{align*}
We have assumed all elements to be fully ionised.

Further, I have four functions that return the right-hand side in eq. \refeq{drdm},
\refeq{dPdm}, \refeq{dLdm} and \refeq{dTdm}. These are called upon in another function
that integrates all the equations. For simplicity, I have chosen the forward Euler
integration scheme as numerical integration method. 

\subsection{The Euler integration scheme}


\begin{thebibliography}{9}
		\bibitem{stix} 
		Michael Stix,
		\emph{The Sun}.
		Springer, New York,
		2nd Edition,
		2002.

		\bibitem{gudiksen}
		Boris Gudiksen
		\emph{AST3310: Astrophysical plasma and stellar interiors}.
		2014.

\end{thebibliography}

\end{document}

